%%%%%%%%%%%%%%%%%%%%%%%%%%%%%%%%%%%%%%%%%%%%%%%%%%%%%%%%%%%%%%%%%%%%%%%%%%%%%%%%
% Medium Length Graduate Curriculum Vitae
% LaTeX Template
% Version 1.2 (3/28/15)
%
% This template has been downloaded from:
% http://www.LaTeXTemplates.com
%
% Original author:
% Rensselaer Polytechnic Institute 
% (http://www.rpi.edu/dept/arc/training/latex/resumes/)
%
% Modified by:
% Daniel L Marks <xleafr@gmail.com> 3/28/2015
%
% Important note:
% This template requires the res.cls file to be in the same directory as the
% .tex file. The res.cls file provides the resume style used for structuring the
% document.
%
%%%%%%%%%%%%%%%%%%%%%%%%%%%%%%%%%%%%%%%%%%%%%%%%%%%%%%%%%%%%%%%%%%%%%%%%%%%%%%%%

%-------------------------------------------------------------------------------
%	PACKAGES AND OTHER DOCUMENT CONFIGURATIONS
%-------------------------------------------------------------------------------

%%%%%%%%%%%%%%%%%%%%%%%%%%%%%%%%%%%%%%%%%%%%%%%%%%%%%%%%%%%%%%%%%%%%%%%%%%%%%%%%
% You can have multiple style options the legal options ones are:
%
%   centered:	the name and address are centered at the top of the page 
%				(default)
%
%   line:		the name is the left with a horizontal line then the address to
%				the right
%
%   overlapped:	the section titles overlap the body text (default)
%
%   margin:		the section titles are to the left of the body text
%		
%   11pt:		use 11 point fonts instead of 10 point fonts
%
%   12pt:		use 12 point fonts instead of 10 point fonts
%
%%%%%%%%%%%%%%%%%%%%%%%%%%%%%%%%%%%%%%%%%%%%%%%%%%%%%%%%%%%%%%%%%%%%%%%%%%%%%%%%

\documentclass[margin]{res}  

% Default font is the helvetica postscript font
\usepackage{helvet}

% Increase text height
\textheight=700pt
\usepackage{fancyhdr}
\pagestyle{fancy}
\renewcommand{\headrulewidth}{0pt}
\fancyhf{}
\rfoot{Page \thepage}

\begin{document}


%-------------------------------------------------------------------------------
%	NAME AND ADDRESS SECTION
%-------------------------------------------------------------------------------
\name{William V. Jardee \vspace{3ex}}

% Note that addresses can be used for other contact information:
% -phone numbers
% -email addresses
% -linked-in profile

\address{2411 Wheeler Dr. Unit D\\Bozeman, MT 59715\\United States of America\vspace{1ex}}
\address{github.com/WillJardee\\ willjardee@gmail.com\\(406) 836-2338}

% Uncomment to add a third address
%\address{Address 3 line 1\\Address 3 line 2\\Address 3 line 3}
%-------------------------------------------------------------------------------

\begin{resume}

%-------------------------------------------------------------------------------
%	EDUCATION SECTION
%-------------------------------------------------------------------------------
\section{EDUCATION}
\raggedright
\textbf{Montana State University}, Bozeman, MT\\\vspace{0.5ex}
{\sl B.S., Summa Cum Laude}, Professional Physics \hfill GPA: 3.94/4.0\\
Phi Kappa Phi Honors Society\\
Minors in: Computer Science, Mathematics,\hfill {\sl (Expected) May 2022}


%-------------------------------------------------------------------------------
%	PROJECTS SECTION
%-------------------------------------------------------------------------------
\section{PROJECTS}
\raggedright
\par
\textbf{Relativistic Runaway Electrons and Lightning Discharge; A Qualitative Overview}: 
A paper on the building blocks of the RREA Theory, alongside motivations, computational and experimental evidence. Survey of step leaders and the related TGF emissions.
\par
\textbf{RREA Propagation Theory; A Theoretical and Computational Overview}:
A delve into the theoretical derivation of RREA theory and the implementation of complete Monte Carlo simulations of particle propagation in storm-clouds. ({\sl Ongoing project})
\par 
\textbf{Introduction to Computational Physics}:
An overview of Python, LaTeX , and other essential tools to computational sciences. The overview covers both fundamental concepts and detailed delves into specific topics. ({\sl Ongoing project})

%-------------------------------------------------------------------------------

%-------------------------------------------------------------------------------
%	COMPUTER SKILLS SECTION
%-------------------------------------------------------------------------------
\section{TECHNICAL\\SKILLS}

{
\begin{tabular}{r p{9.5cm}}
\textbf{Languages:}   & Python, Java, C/C++, Matlab, Mathematica, GitHub, LaTeX, HTML, CSS, Excel \vspace{0.5ex}\\
\textbf{Mathematics:} & Linear Algebra, Dynamical/Chaotic Systems, Computation Theory \vspace{0.5ex} \\
\textbf{Physics:}     & Particle Physics, Observational Astronomy   \\
\end{tabular}
}



%-------------------------------------------------------------------------------

%-------------------------------------------------------------------------------
%	EXPERIENCE SECTION
%-------------------------------------------------------------------------------
% Modify the format of each position
\begin{format}
\title{l}\employer{r}\\
\dates{l}\location{r}\\
\body\\
\end{format}
%-------------------------------------------------------------------------------

\section{TEACHING EXPERIENCE}
\noindent
\textbf{Hillman Scholars Tutor}\\
{\sl Allen Yarnell Student Success Center, Montana State University, Bozeman}\\\vspace{0.5ex}
Educated underprivileged college students in introductory math, physics, computer science, and humanities courses. \hfill
{\sl July~2021~-~Present}

%---
\noindent
\raggedright
\textbf{Math Stats Center Tutor}\\
{\sl Mathematics Department, Montana State University, Bozeman}\\\vspace{0.5ex}
Guided students to discover their own answers and understanding in classes ranging from introductory algebra to differential equations. \hfill
{\sl Aug~2021~-~Present}

%---
\noindent
\raggedright
\textbf{Proctor/Grader (PHSX 207)}\\
{\sl Physics Department, Montana State University, Bozeman}\\\vspace{0.5ex}
Graded weekly homework and exams of algebra based introductory physics course.\\
\raggedleft {\sl Jan~2021~-~May~2021}

%---
\noindent
\raggedright
\textbf{Student Lab Assistant (PHSX 205)}\\
{\sl Physics Department, Montana State University, Bozeman}\\\vspace{0.5ex}
Guided students of the introductory physics course through kinematic labs during a weekly lab. \hfill
{\sl Aug~2020~-~Nov~2020}

%---
\noindent
\raggedright
\textbf{Smarty Cats Tutor}\\\vspace{0.5ex}
{\sl Allen Yarnell Student Success Center, Montana State University, Bozeman}\\
Made and personalized appointments with student to cover essential STEM topics.\\
\raggedleft {\sl Aug~2019~-~May~2020}

%---
\noindent
\raggedright
\textbf{Volunteer STEM Tutor}\\
{\sl The Rock Youth Center, Bozeman, MT}\\\vspace{0.5ex}
Held open tutoring hours for high school students who struggle in STEM subjects.\\
\raggedleft
{\sl Oct~2019~-~March~2020}

%-------------------------------------------------------------------------------

\section{RESEARCH EXPERIENCE}
\noindent
\raggedright
\textbf{Undergraduate Researcher}\\
{\sl Dr. John Sample's Lab, Montana State University, Bozeman}\\\vspace{0.5ex}
Analyzed the performance of a soft x-ray spectrometer to be attached to the IMPRESS CUBE-SAT.  \hfill
{\sl Aug~2020~-~Dec~2020}

%---
\noindent
\raggedright
\textbf{Undergraduate Researcher}\\
{\sl Dr. Rufus Cone's Lab, Montana State University, Bozeman}\\\vspace{0.5ex}
Studied the theory behind and attempted to use an ellipsometer to measure the thickness of thin wafers.  \hfill
{\sl Jan~2020~-~Apr~2020}

%-------------------------------------------------------------------------------

\section{MISC EXPERIENCE}
\noindent
\raggedright
\textbf{\textbf{SPS Treasurer}}\\
{\sl Society of Physics Students at Montana State University, Bozeman}\\\vspace{0.5ex}
Handled club finances and lead many efforts in stirring interest in science communication and computational physics.   \hfill
{\sl Feb~2020~-~Jan~2022}

%---
\noindent
\raggedright
\textbf{Advanced Physics Lab}\\
{\sl Instructed Course, Montana State University, Bozeman}\\\vspace{0.5ex}
Used sophisticated particle trapping apparatuses and computational methods to measure the mass of picogram scale particles.  \hfill
{\sl Aug~2021~-~Dec~2021}

%-------------------------------------------------------------------------------
%	Awards
%-------------------------------------------------------------------------------
\section{AWARDS/ GRANTS}
\textbf{Physics Departmental Scholarship}\\
\hspace{3ex} {\sl Norman Mac Rugheimer Scholarship} \hfill {\sl Aug 2021}\\\vspace{0.5ex}
\hspace{3ex} {\sl Asbridge Physics Scholarship} \hfill {\sl Aug 2020}\\\vspace{0.5ex}
\textbf{Montana University Systems Scholarship}\hfill\hfil {\sl May 2018}\\\vspace{0.5ex}
\textbf{Bertha Feaster Scholarship} \hfill {\sl May 2018}

%-------------------------------------------------------------------------------
%	Presentations
%-------------------------------------------------------------------------------
\section{POSTERS/ PRESENTATIONS}

\textbf{SPS Undergraduate Colloquium}\\
\hspace{3ex} {\sl RREA Propagation Theory} \hfill {\sl Oct~2021}\\\vspace{0.5ex}
\hspace{3ex} {\sl The Better Poster Design} \hfill {\sl Feb~2021}\\\vspace{0.5ex}
\hspace{3ex} {\sl Teaching Yourself Computer Languages} \hfill {\sl Feb~2021}\\\vspace{0.5ex}
\hspace{3ex} {\sl Introduction to Python} \hfill {\sl Feb~2021}\\\vspace{0.5ex}
\hspace{3ex} {\sl The Basics of Climate Physics} \hfill {\sl Sept~2020}\\

	
%-------------------------------------------------------------------------------
%	Outreach
%-------------------------------------------------------------------------------
\section{OUTREACH}
\textbf{Museum of the Rockies}\\
\hspace{3ex} {\sl Grossology}\hfill {\sl Oct~2021}\\\vspace{0.5ex}
\textbf{Society of Physics Students}\\
\hspace{3ex} {\sl Liquid Nitrogen Ice Cream}\hfill {\sl Oct~2021}\\\vspace{0.5ex}
\hspace{3ex} {\sl Careers in Industry Panel; Moderator}\hfill {\sl ~Mar~2021,~Oct~2020}\\


%-------------------------------------------------------------------------------
%	Interests
%-------------------------------------------------------------------------------
\section{INTERESTS}
Science communication, computational physics, particle physics, chaotic systems, RREA propagation theory

%-------------------------------------------------------------------------------
\end{resume}
\end{document}