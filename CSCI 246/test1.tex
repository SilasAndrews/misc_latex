\documentclass[11pt]{article}

\usepackage{amssymb,amsmath}
\usepackage{times,psfrag,epsf,epsfig,graphics,graphicx,caption}
\usepackage{enumitem}
\usepackage{algorithm}
\usepackage{algorithmic}

\begin{document}
\date{}

\title{CSCI 246: Test 1}

\author{William Jardee}

\maketitle


\section*{Problem 1}
\begin{center}
    Let $S = \{a,b,1,2\}$. List the power set (set of all the subsets of S).
\end{center}
    \[\mathcal{P}(S) = \{\{\},\{a\},\{b\},\{1\},\{2\},\{a,b\},\{a,1\},\{a,2\},\{b,1\},\{b,2\},\{1,2\},\]
    \[\{a,b,1\},\{b,1,2\},\{a,b,2\},\{a,1,2\},\{a,b,1,2\}\}\]
    
\newpage

\section*{Problem 2}
\begin{center}
    If $x$ is divisible by 18, then $x$ is divisible by 9 and $x$ is divisible by 2.\\
\end{center}
    $p = x$ is divisible by 18\\
    $q = x$ is divisible by 9 $\and x$ is divisible by 2
\begin{enumerate}
    \item Negation:\\
        $x$ is divisible by 18 and [$x$ is not divisible by 9 or $x$ is not divisible by 2]\\\\
        Alternatively you could write:\\
        $x$ is divisible by 18 and $x$ is not divisible by 9 or $x$ is divisible by 18 and $x$ is not divisible by 2
    \item Contrapositive:\\
        If $x$ is not divisible by 9 or $x$ is not divisible by 2, then $x$ is not divisible by 18.
    \item Converse:\\
        If $x$ is divisible by 9 and $x$ is divisible by 2, then $x$ is divisible 18.
    \item Inverse:\\
        If $x$ is not divisible by 18, then $x$ is not divisible by 9  or $x$ is not divisible by 2.

\end{enumerate}

\newpage

\section*{Problem 3}
    \quad If a computer program is correct, then compilation of the program does not produce error messages.\\\\
    Compilation of this program produces error messages. \\
    $\therefore$ The computer program is not correct.\\\\
    I just needed to finish the contrapositive.

\newpage

\section*{Problem 4}
\begin{center}
    What are A and B?
\end{center}
This is a perfectly logical statement. If A is a knight and B is a Knave, then A is correct in saying that B is a Knave, as B is incorrect. I would say, however, that these statements by A and B are confusing. \\
\begin{center}
    A is a knight \quad B is a Knave
\end{center}

\newpage

\section*{Problem 5}
\begin{enumerate}
    \item Write a negation for the statement:\\
    \quad $\exists$ integer $d$ if  $\frac{18}{d}$ is an integer, then $d = 3$.\\
    Negation: $\forall$ integer $d$, $\frac{18}{d}$ is an integer and $d \neq 3$. 
    \item Is the statement in 5.1 correct? Explain your reason.\\
    It took me a couple times looking over this question to actually get the right answer eventually. \\
    $\exists$ integer $d$ if  $\frac{18}{d}$ is an integer, then $d = 3$;\\
    Since $d=3$ is an integer, and $\frac{18}{3} = 6$, $6$ is an integer so the statement is valid (correct).


\end{enumerate}

\end{document}
