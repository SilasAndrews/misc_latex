\documentclass[11pt]{article}

%Don't change any thing before \begin{document}
%They are not useful for now, but later when you try to add figures
%these might be useful. In fact if you use sth fancy, you might need
%to add more packages, or macros.
\usepackage{amssymb,amsmath}
\usepackage{times,psfrag,epsf,epsfig,graphics,graphicx}
\usepackage{algorithm}
\usepackage{algorithmic}

\begin{document}
\date{}

\title{CSCI 246: Assignment~6~(6 points)}

\author{William Jardee}

\maketitle

 
\section*{Problem 1.}

\noindent
Given real-valued functions $f$ and $g$ with the same domain, the sum of $f$ and $g$, denoted as $f+g$, is defined as follows.
\newline

For each real number $x$, $(f+g)(x)=f(x)+g(x)$.

\noindent
Prove that if $f$ and $g$ are both increasing on a set $S$ of real numbers,
then $f+g$ is also increasing on $S$.
\newline
\newline
\noindent
{\bf Proof:}~~
\newline

Assume that $f$ and $g$ are increasing on $S$. Let $x_1 < x_2$ and $x_1, x_2 \in S$. Since $f$ and $g$ and increasing we can say that $f(x_1)<f(x_2)$ and $g(x_1)<g(x_2)$. Then the following occurs:
\[f(x_1)+g(x_1) < f(x_1) + g(x_2)\]
\[(f+g)(x_1) < (f+g)(x_2)\]
This is valid for all $x_1$ and $x_2$ in $S$, so we can say that $f+g$ is increasing on all $S$.
\begin{flushright}$\blacksquare$\end{flushright}
\newpage

\section*{Problem 2.}

\noindent
Given a polynomial $f(n)=\frac{1}{4}n^2-24n-16$, show the following:
\newline

\noindent
(2.1) Show that $f(n)=\frac{1}{4}n^2-24n-16$ is $\Omega(n^2)$.\\
\noindent
{\bf Answer:}~~
Following the procedure given in class, we get that 
\begin{center}
    $\frac{1}{8}n^2 \leq \frac{1}{4}n^2 -24n -16$ for all $n\geq 320$.
\end{center}
To verify this, we first check $n=320$, the equality gives $12800 \leq 17904$, which is a valid statement. So  next we have to show that for $n\geq 320$, the right side grows faster than the left. We will use some basic calculus to make this explanation easier. Taking the first derivative yields:
\begin{center}
    $\frac{1}{4}n$ on the right, and $\frac{1}{2}n - 24$  on the left
\end{center}
Obviously for $n\geq 320$, the left sided first derivative will be less than the left, so the left will grow faster and always stay ahead of the right for at least $n\geq 320$. We have met all the conditions to say that $f(n)$ is $\Omega(n^2)$.\\\\

\noindent
(2.2) Show that $f(n)=\frac{1}{4}n^2-24n-16$ is $O(n^2)$.\\
\noindent
{\bf Answer:}~~
It is enough to say that $\frac{1}{4}n^2 -24n -16 \geq \frac{1}{4}n$ for $n>0$. So, $f(n)$ is $O(n^2)$.\\\\

\noindent
(2.3) Show that $f(n)=\frac{1}{4}n^2-24n-16$ is $\Theta(n^2)$.\\
\noindent
{\bf Answer:}~~
To classify a function as $\Theta(n^2)$ it must be both $\Omega(n^2)$ and $O(n^2)$. We have shown that both those conditions are met, so we can say that $f(n)$ is $\Omega(n^2)$.

\newpage

\section*{Problem 3.}

Given three positive real-valued functions $f_1,f_2$ and $g$ defined on the
same set of nonnegative integers.  Prove that if
$f_1(n)$ is $\Theta(g(n))$ and $f_2(n)$ is $\Theta(g(n))$, then
$F(n)=f_1(n)+f_2(n)$ is $\Theta(g(n))$.\\\\
\noindent
{\bf Proof:}~~ 
\newline
Take the conditions that $f_1(n)$ is $\Theta(g(n))$, $f_2(n)$ is $\Theta(g(n))$, and $F(n) = f_1(n) + f_2(n)$. In order to say that $F(n)$ is $\Theta(g(n))$, $F(n)$ must be $\Omega(g(n))$ and $O(g(n))$. Since $f_1(n)$ is $\Theta(g(n))$ and $f_2(n)$ is $\Theta(g(n))$, we can say: 
\begin{center}
    $f_1(n) > a\cdot g(n)$, $f_1(n) < c\cdot g(n)$, $f_2(n) > b\cdot g(n)$, and $f_2(n) < d\cdot g(n)$, s.t. $a,b,c,d \in \mathbb{R}$
\end{center}
\[a \cdot g(n) + b \cdot g(n) < f_1(n) + f_2(n) < c \cdot g(n) + d \cdot g(n)\]
\[(a+b) \cdot g(n) < F(n) < (c+d) \cdot g(n)\]
So in one statement we say that $F(n)$ is $\Omega (g(n))$ and $O(g(n))$. Thus $F(n)$ is $\Theta (g(n))$.
\begin{flushright}$\blacksquare$\end{flushright}
\newpage

\section*{Problem 4.}

When analyzing the average running time for Insertion Sort, we have the
following recurrence relation:

$E_k=E_{k-1}+\frac{k+1}{2}$, for each integer $k\geq 2$,

$E_1=0$.
\newline

(4.1) Use the direct iteration method to compute $E_n$. Sufficient details must be given.\\
\noindent
{\bf Answer:}~~ 
\newline

$E_1 = 0$\\
$E_2 = \frac{3}{2}$\\
$E_3 = \frac{3}{2} + \frac{4}{2}$\\
$E_4 = \frac{3}{2} + \frac{4}{2} + \frac{5}{2}$\\\\
\noindent
$E_n = \frac{\sum\limits^{n+1}_3 (i)}{2} = \frac{1}{2}\Big(\frac{(n+1)(n+2)}{2}-3\Big)$\\\\

\noindent
(4.2) Use mathematical induction to verify the correctness of the formula.\\\\
\noindent
{\bf Proof by Induction:}~~ 
\newline
Base case: $n=1$; $E_1 = \frac{1}{4}(2\cdot3 - 6) = 0$. So the base case works.\\
Inductive step: We say that the equation works for $k$, so we need to show that $k+1$ works.\\
\[E_{k+1} = \frac{1}{4}((k+2)(k+3)-6)\]
\[E_{k+1} = \frac{1}{4}((k^2+5k+6)-6)\]
\[E_{k+1} = \frac{1}{4}((k^2+3k+3)+2k+4-6)\]
\[E_{k+1} = \frac{1}{4}((k+1)(k+2)-6) +\frac{2k+4}{4}\]
\[E_{k+1} = E_{k} + \frac{(k+1)+1}{2}\]\\
So the equation holds for $k+1$. By induction, $E_n = \frac{1}{2}\Big(\frac{(n+1)(n+2)}{2}-3\Big)$ is correct.
\begin{flushright}$\blacksquare$\end{flushright}

\newpage


\section*{Problem 5.}

Prove that $\lfloor\log_2n\rfloor$ is $\Theta(\log_2n)$.\\

\noindent
{\bf Proof:}~~\\
In order to prove that $\lfloor\log_2n\rfloor$ is $\Theta(\log_2n)$, we must show that $\lfloor\log_2n\rfloor$ is $\Omega(\log_2n)$ and $\lfloor\log_2n\rfloor$ is $O(\log_2n)$. Let us first show that $\lfloor\log_2n\rfloor$ is $\Omega(\log_2n)$. Since the floor operator rounds down, $\lfloor \log_2n \rfloor \geq \log_2n -1$. Looking to calculus to provide an insight about the limits; $\lim_{x \to +\infty} \log_2n \rightarrow \infty$. So, as $\log_2n$ tends to large $n$, the -1 can be neglected and we can say that $\lfloor \log_2n \rfloor \geq a\log_2n$, s.t. $a<1$ if $n$ is significantly large. So, we have met the condition to say $\lfloor\log_2n\rfloor$ is $\Omega(\log_2n)$. Now to show that $\lfloor\log_2n\rfloor$ is $O(\log_2n)$. Since the floor operator rounds down, $\lfloor \log_2n \rfloor \leq \log_2n < b\log_2n$ if $b>1$. So, we can write $\lfloor \log_2n \rfloor < b\log_2n$. So, $\lfloor\log_2n\rfloor$ is $O(\log_2n)$. Since we have shown $\lfloor\log_2n\rfloor$ is $\Omega(\log_2n)$ and $\lfloor\log_2n\rfloor$ is $O(\log_2n)$, $\lfloor\log_2n\rfloor$ is $\Theta(\log_2n)$.
\begin{flushright}$\blacksquare$\end{flushright}


\end{document}

