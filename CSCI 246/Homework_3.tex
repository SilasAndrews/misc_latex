\documentclass[11pt]{article}

%Don't change any thing before \begin{document}
%They are not useful for now, but later when you try to add figures
%these might be useful. In fact if you use sth fancy, you might need
%to add more packages, or macros.
\usepackage{amssymb,amsmath,algorithm, algorithmic, mathtools}
\usepackage{times,psfrag,epsf,epsfig,graphics,graphicx,caption,enumitem}

\DeclarePairedDelimiter\ceil{\lceil}{\rceil}
\DeclarePairedDelimiter\floor{\lfloor}{\rfloor}

\begin{document}
\date{}

\title{CSCI 246: Assignment~3~(6 points)}

\author{William Jardee}

\maketitle


\noindent
This assignment is due on {\bf Thursday, Oct 1, 8:30pm}. It is strongly
encouraged that you use Latex to generate a single pdf file and upload it
under {\em Assignment 3} on D2L. But there will NOT be a penalty for not
using Latex (to finish the assignment). This could be a group-assignment,
so you could form a group with $\leq 3$ students (mathematically, this means
you can also do it by yourself).
\newline
 
\section*{Problem 1.}

\noindent
Prove that if $a$ and $b$ are odd integers, then $a^2+b^2$ is even.
\newline

Suppose we have two odd integers that, by the definition of odd, $a=2n+1$ and $b = 2k+1$ s.t. $n,k \in \mathbb{Z}$. Then;
\begin{center}  $a^2 = 4n^2 +  4n + 1$ and $b^2 = 4k^2 + 4k + 1$\end{center}
\[a^2 + b^2 = 4n^2 +4n +1 + 4k^2 + 4k +1 = 2(2n^2 +2n +2k^2 +2k +1) = 2q\]
s.t. $q = 2n^2 +2n +2k^2 +2k +1$. By closure of integers over multiplication and addition $q \in \mathbb{Z}$. By the definition of even, we can then say that $a^2 + b^2$ is even. 
\begin{flushright}$\blacksquare$\end{flushright}

\newpage


\section*{Problem 2.}

\noindent
Reorder the premises in the argument to show that the conclusion follows as
a valid consequence from the premises. It might be helpful to rewrite the
statements in if-then form and replace some of them by their contrapositives.
{\em You can assume the premise: ``The arguments in these examples are not arranged in regular order like the ones I am used to".}
\newline

{\bf A}. When I work a logic example without grumbling, you may be sure it is one I understand.
\newline

{\bf B}. No easy examples make my head ache.
\newline

{\bf C}. I can't understand examples if the arguments are not arranged in regular order like the ones I am used to.
\newline

{\bf D}. I never grumble at an example unless it gives me a headache.
\newline

$\therefore$~~~ These examples are not easy.
\\\\

{\em Reworded logic:}\\

{\bf A}. If I work a logic example without grumbling, then I understand it.\\

{\bf B}. If it an easy example, then my head doesn't ache.\\

{\bf C}. If the arguments are not arranged in regular order like the ones I am used to, then I can't understand the examples.\\

{\bf D}. I grumble at an example if and only if it gives me a headache.\\

$\therefore$ These examples are not easy
\\\\

{\em Reordering the logic:}\\
{\bf Original Premise:} The arguments in these examples are not arranged in regular order like the ones I am used to.\\

{\bf C} If the arguments are not arrange in regular order liek the ones I am used to, then I can't understand the examples.\\

{\bf A} If I do not understand the example, I must work the logic example with grumbling.\\

{\bf D} I grumble at an example if and only if it gives me a headache.\\

{\bf B} If my head aches, then it isn't an easy example.\\

$\therefore$ These examples are not easy

\newpage

\section*{Problem 3.}

Sam worked hard last night; in fact, he slept for only three hours. He claimed that he found the following theorem:

For every integer $n>3$, if $n$ is even then $n^2+1$ is prime.
\newline

\noindent
Find a counterexample for Sam's claim.
\newline

If $n =4$ then $n$ is obviously even and $n^2 + 1 = 65$ which is divisible by 5, violating the definition of prime. So we see a contradiction with the original statement. \\

$\boxed{n  = 4}$

%\noindent
%{\bf Answer:}~~ .........
%\newline

\newpage

\noindent
\section*{Problem 4.}

Prove the following statement directly by definition.
\newline

The difference of any two rational numbers is a rational number.
\newline

Proof:

Let a and b be rational numbers. By the definition of rational numbers, a and b can be written as $a = \frac{c}{d}$ and $b = \frac{e}{f}$ s.t. $c,d,e,f \in \mathbb{Z}$. 
\[a - b = \frac{c}{d} - \frac{e}{f} = \frac{cf -ed}{df} = \frac{k}{q}\]
s.t. $k = cf -ed$ and $q = df$. By the closer of integers over multiplication and addition, we can say that $k,q \in \mathbb{Z}$. By the definition of rational numbers, we can say that $a-b$ is a rational number. But $a$ and $b$ were arbitrary rational numbers, so the statement can be generalized that the difference of any two rational numbers is a rational number.
\begin{flushright}$\blacksquare$\end{flushright}

\newpage


\section*{Problem 5.}

If $n$ is an integer and $n>1$, then $n!$ is the product of $n$ and every other positive integer that is less than $n$, e.g., $5!=5\times 4\times 3\times 2\times 1$.
\newline

\noindent
(5.1) Write 6! in standard factored form.\\
Using De Polignac's Formula for the primes less than 20: 
$$a_n = \sum_{k>0} \floor*{\frac{20}{n^k}}$$
$a_2 = 4$\\
$a_3 = 2$\\
$a_5 = 1$\\
\[\boxed{6! = 2^4 \times 3^2 \times 5}\]

\noindent
(5.2) Write 20! in standard factored form.\\
Using De Polignac's Formula for the primes less than 20: 
$$a_n = \sum_{k>0} \floor*{\frac{20}{n^k}}$$
$a_2 = 18$\\
$a_3 = 8$\\
$a_5 = 4$\\
$a_7 = 2$\\
$a_{11} = a_{13} = a_{17} = a_{19} = 1$
\[\boxed{20! = 2^{18} \times 3^8 \times 5^4 \times 7^2 \times 11 \times 13 \times 17 \times 19}\]


\noindent
(5.3) Without computing the value of $(20!)^2$, determine how many zeros are at the end of the number when it is written in decimal form. Explain the reason.\\

\[20!^2 = (2^{18} \times 3^8 \times 5^4 \times 7^2 \times 11 \times 13 \times 17 \times 19)^2\]
\[20!^2 = 2^{36} \times 3^{16} \times 5^8 \times 7^4 \times 11^2 \times 13^2 \times 17^2 \times 19^2\]
Each zero at the end can be factored out as a multiple of 10. 10 has a prime factorization of $10 = 2 \times 5$, so there are only as many 10's are there are 2's or 5's, whichever there are fewer. So there are $\boxed{8}$ zeros at the end of $20!^2$.

\end{document}
