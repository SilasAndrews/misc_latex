\documentclass[11pt]{article}

\usepackage[english]{babel}
\usepackage[margin=0.8in]{geometry}

% Math/Greek packages
\usepackage{amssymb,amsmath,amsthm, mathtools} 
\usepackage{algorithm, algorithmic}
\usepackage{upgreek, siunitx}
\usepackage{setspace}

% Graphics/Presentation packages
\usepackage{multirow}
\usepackage{graphicx}
\usepackage{cancel}
\usepackage{tabulary, enumitem, array}
\usepackage{xparse,mleftright,tikz}
\usepackage{physics}

% Misc packages
\usepackage{fancyhdr}


\usepackage[export]{adjustbox}

\usepackage{esint}

\sisetup{locale=US,group-separator = {,}}
\usepackage[colorlinks=true, allcolors=blue]{hyperref}


% General macro declarations


\makeatletter
\let\oldabs\abs
\def\abs{\@ifstar{\oldabs}{\oldabs*}}
%
\let\oldnorm\norm
\def\norm{\@ifstar{\oldnorm}{\oldnorm*}}
\makeatother

\begin{document}

\title{PHSX 491: HW08}
\author{William Jardee}
\maketitle

\section*{Question 1}
\begin{enumerate}[label=\alph*)]
\item \textit{Show that } $\displaystyle{\Gamma^\lambda_{\mu\nu} = 0}$.

Let us begin with the equation of the Christoffel symbol
\[\Gamma^\lambda_{\nu \mu} = \frac{1}{2}g^{\beta \lambda}\left[\partial_\mu g_{\nu \beta} + \partial_\nu g_{\mu \beta} - \partial_\beta g_{\nu\mu}\right].\]
Taking the fact that the metric we are looking at is diagonal, then the equation simplifies as
\begin{align*}
\Gamma^\lambda_{\nu \mu} &= \frac{1}{2}g^{\beta \lambda}\left[\partial_\mu g_{\nu \beta} + \partial_\nu g_{\mu \beta} - \partial_\beta g_{\nu\mu}\right]\\
& = \frac{1}{2}g^{\lambda\lambda} \left[\cancelto{0}{\partial_\mu g_{\nu \lambda}} + \cancelto{0}{\partial_\nu g_{\mu \lambda}} - \cancelto{0}{\partial_\lambda g_{\nu\mu}}\right]\\
& = \, \boxed{0} \, .  \qquad \checkmark
\end{align*}

\item \textit{Show that } $\displaystyle{\Gamma^\lambda_{\mu \mu} = -\frac{1}{2}\left(g_{\lambda\lambda}\right)^{-1}\partial_\lambda \, g_{\mu\mu}}$.

This starts at the same point, but not everything cancels out now:
\begin{align*}
\Gamma^\lambda_{\mu\mu} & = \frac{1}{2}g^{\beta \lambda} \left[\partial_\mu g_{\mu\beta} + \partial_\mu g_{\mu \beta} - \partial_\beta g_{\mu \mu}\right]\\
& = \frac{1}{2} g^{\lambda\lambda} \left[\cancelto{0}{\partial_\mu g_{\mu \lambda}} + \cancelto{0}{\partial_\mu g_{\mu \lambda}} - \partial_\lambda g_{\mu\mu}\right]\\
& = \frac{1}{2}g^{\lambda\lambda}\left[-\partial_\lambda g_{\mu\mu}\right]\\
& = \, \boxed{-\frac{1}{2}\left(g_{\lambda\lambda}\right)^{-1}\partial_\lambda \, g_{\mu\mu}} \, \qquad \checkmark
\end{align*}
where the last line is justified by realizing that the metric is diagonal and, thus, the inverse is the reciprocal of the diagonal elements.

\item \textit{Show that } $\displaystyle{\Gamma^\lambda_{\mu \lambda} = \partial_\mu \left(\ln(\sqrt{\abs{g_{\lambda\lambda}}})\right)}$.

This one is quite clever, I like it! Let's start where we have started the last two parts (but skipping a couple cancellations in the beginning):
\begin{align*}
\Gamma^\lambda_{\mu \lambda} & = \frac{1}{2} g^{\lambda\lambda}\left[\cancelto{0}{\partial_\lambda g_{\mu\lambda}} + \partial_\mu g_{\lambda\lambda}\right]\\
& = \frac{1}{2}\left(g_{\lambda\lambda}\right)^{-1}\partial_\mu \, g_{\lambda\lambda}\\
& = \frac{1}{2}\left(g_{\lambda\lambda}\right)^{-1/2}\left(g_{\lambda\lambda}\right)^{-1/2}\partial_\mu g_{\lambda\lambda}\\
& = \, \boxed{\partial_\mu \left(\ln(\sqrt{\abs{g_{\lambda\lambda}}})\right)} \, . \qquad \checkmark
\end{align*}

\item \textit{Show that } $\displaystyle{\Gamma^\lambda_{\lambda \lambda} = \partial_\lambda \left(\ln(\sqrt{\abs{g_{\lambda\lambda}}})\right)}$.

If we take the previous part and let $\mu \rightarrow \lambda$ then we get to 
\[\Gamma^\lambda_{\lambda \lambda} = \partial_\lambda \left(\ln(\sqrt{\abs{g_{\lambda\lambda}}})\right).\]
This, however, does overlook some of the cancellations of $g_{\mu \lambda}$. So, let's recalculate the beginning of the derivation.
\begin{align*}
\Gamma^\lambda_{\lambda\lambda} & = \frac{1}{2}g^{\lambda\beta} \left[\partial_\lambda g_{\lambda\beta} + \partial_\lambda g_{\lambda\beta} - \partial_\beta g_{\lambda\lambda}\right]\\
& = \frac{1}{2}g^{\lambda\lambda} \left[\partial_\lambda g_{\lambda\lambda} + \partial_\lambda g_{\lambda\lambda} - \partial_\lambda g_{\lambda\lambda}\right]\\
& = \frac{1}{2}g^{\lambda\lambda} \left[\partial_\lambda g_{\lambda\lambda}\right]
\end{align*}
and, at this point, we have gotten back to the calculation done in the previous part. So, we will just jump to the conclusion of 
\[\Gamma^\lambda_{\lambda\lambda} = \, \boxed{\partial_\lambda \left(\ln(\sqrt{\abs{g_{\lambda\lambda}}})\right)} \, .\]

\item \textit{Find the Christoffel symbols for the Schwarzschild spacetime}.

There are a total of nine that turn out to be non-negative (13 if you count identical ones), those are:
\begin{align*}
\Gamma^t _{rt} = \Gamma^t_{tr} & = \partial_r \left(\ln{\sqrt{\abs{g_{tt}}}}\right)\\
& = \left(1 - \frac{2GM}{r}\right)^{-1}\left(\frac{2GM}{r^2}\right)\\
\Gamma^r_{rr} &= \left(1 - \frac{2GM}{r}\right)^{-1}\left(\frac{2GM}{r^2}\right)\\
\Gamma^r_{\theta\theta} & = -\frac{1}{2}\left(1-\frac{2GM}{r}\right)2r\\
& = -(r-2GM)\\
\Gamma^r_{\phi\phi} & = -(r-2GM)\sin[2](\theta)
\end{align*}
\begin{align*}
\Gamma^\theta_{\theta r} = \Gamma^\theta_{r \theta} & = \partial_r \, \ln(\sqrt{r^2}) = \frac{1}{r}\\
\Gamma^\theta_{\phi \phi} & = -\frac{1}{2}\left(\frac{1}{r^2}\right)\partial_\theta \, r^2 \sin[2](\theta)\\
& = -\sin(\theta)\cos(\theta)\\
\Gamma^\phi_{\theta \phi} = \Gamma^\phi_{\phi \theta} & = \partial_\theta \, \ln(r^2\sin[2](\theta))\\
& = \frac{1}{\sin[2](\theta)} \sin(\theta)\cos(\theta) = \cot(\theta)\\
\Gamma^r_{tt} & = \frac{GM}{r^2}\left(1 - \frac{2GM}{r}\right)\\
\Gamma^\phi_{\phi r} = \Gamma^\phi_{r \phi} & = \partial_r \, \ln(r\sin(\theta)) = \frac{1}{r} \, .
\end{align*}

Wow, that's a lot of mess. Let's state them as (removing the duplicates) and letting $\theta = \pi/2$:
\begin{align*}
\Gamma^t_{rt} & = \left(1 - \frac{2GM}{r}\right)^{-1}\left(\frac{2GM}{r^2}\right) & \Gamma^r_{rr} & = \left(1 - \frac{2GM}{r}\right)^{-1}\left(\frac{2GM}{r^2}\right) & \Gamma^r_{\theta\theta} & = -(r-2GM)\\
\Gamma^r_{tt} & = \frac{GM}{r^2}\left(1 - \frac{2GM}{r}\right) & \Gamma^r_{\phi\phi} & = -(r-2GM) & \Gamma^\theta_{r \theta} & = \frac{1}{r} \\
 \Gamma^\theta_{\phi \phi} & = -\sin(\theta)\cos(\theta) = 0 & \Gamma^\phi_{\theta \phi} & = \cot(\theta) = 0 & \Gamma^\phi_{r \phi} & = \frac{1}{r} \, .
\end{align*}

The rest of the Christoffel symbols are zero.

\item \textit{Verify your answer to part \textbf{d} with the geodesic equation}.

So, I found it easier to rederive some of the values instead of manipulating our new equations into the form discussed in class. I start with the previous geodesics equation 
\[\dv{\tau} \left(g_{\alpha \mu}\dv{x^\mu}{\tau}\right) - \frac{1}{2}\partial_\alpha g_{\mu\nu}\dv{x^\mu}{\tau}\dv{x^\nu}{\tau} = 0\]
and derive the appropriate part in parallel to the derivation from the equivalent geodesics equation 
\[\dv[2]{x^\mu}{\tau} + \Gamma^\mu_{\alpha \beta}\dv{x^\alpha}{\tau}\dv{x^\beta}{\tau} = 0 \, .\]

I will be skipping quite a few steps on the former (right) derivations for brevity's sake and both of our sanities'.

For $t$:
\begin{align*}
\dv[2]{t}{\tau} - \left(1 - \frac{2GM}{r}\right)^{-1}\left(\frac{2GM}{r^2}\right)\dv{x^t}{\tau}\dv{x^r}{\tau} & = 0 & \dv{\tau}\left(g_{tt}\dv{x^t}{\tau}\right) - \frac{1}{2}\partial_\alpha g_{\mu\nu}\dv{x^\mu}{\tau}\dv{x^\nu}{\tau} & = 0\\
& & \dv{t}{\tau} \dv{\tau}g_{tt} + g_{tt}\dv[2]{t}{\tau} - 0 & = 0\\
& & -\left(1-\frac{2GM}{r}\right)^{-1}\left(\frac{2GM}{r^2}\right)\dv{t}{\tau}\dv{r}{\tau} + \dv[2]{t}{\tau} & = 0
\end{align*}

For $\theta$:
\begin{align*}
\dv[2]{x^\theta}{\tau} + \frac{2}{r}\dv{x^r}{\tau}\dv{x^\theta}{\tau} & = 0 & \dv{\tau}\left(g_{\theta\theta}\dv{\theta}{\tau}\right)-\frac{1}{2}\partial_\theta g_{\mu\nu}\dv{x^\mu}{\tau}\dv{x^\nu}{\tau} & = 0\\
& & \dv[2]{\theta}{\tau}r^2 + \dv{\theta}{\tau}2r \dv{r}{\tau} & = 0 \\
& & \dv[2]{\theta}{\tau} + \frac{2}{r}\dv{\theta}{\tau}\dv{r}{\tau} & = 0 
\end{align*}

For $\phi$:
\begin{align*}
\dv[2]{\phi}{\tau} + \frac{2}{r}\dv{\phi}{\tau}\dv{r}{\tau} & = 0 & \dv{\tau}\left(g_{\phi\phi}\dv{\phi}{\tau}\right)-\frac{1}{2}\partial_\phi g_{\mu\nu}\dv{x^\mu}{\tau}\dv{x^\nu}{\tau} & = 0\\
& & \dv[2]{\phi}{\tau}r^2 + \dv{\phi}{\tau}\dv{r}{\tau}2r & = 0\\
& & \dv[2]{\phi}{\tau} + \frac{2}{r}\dv{\phi}{\tau}\dv{r}{\tau} & = 0
\end{align*}

For $r$:
\begin{align*}
\dv[2]{r}{\tau} + \left(1 - \frac{2GM}{r}\right)^{-1}\left(\frac{2GM}{r}\right)\left(\dv{r}{\tau}\right)^2 & + \left(\frac{GM}{r^2}\right)\left(1-\frac{2GM}{r}\right)\left(\dv{t}{\tau}\right)^2 \\
& - (r - 2GM)\left[\left(\dv{\theta}{\tau}\right)^2 + \left(\dv{\phi}{\tau}\right)^2\right] = 0
\end{align*}

\begin{align*}
\dv{\tau}\left(g_{rr}\dv{r}{\tau}\right)-\frac{1}{2}\partial_r g_{\mu\nu}\dv{x^\mu}{\tau}\dv{x^\nu}{\tau} & = 0\\
\dv[2]{r}{\tau}g_{rr} + \dv{r}{\tau}\dv{\tau}g_{rr} - \frac{1}{2}\left[\partial_r g_{tt}\left(\dv{t}{\tau}\right)^2 + \partial_r g_{rr}\left(\dv{r}{\tau}\right)^2 + \partial_\theta g_{\theta\theta}\left(\dv{\theta}{\tau}\right)^2 + \partial_\phi g_{\phi\phi}\left(\dv{\phi}{\tau}\right)^2\right] & = 0\\
\dv[2]{r}{\tau} + \left(1 - \frac{2GM}{r}\right)^{-1}\left(\frac{2GM}{r}\right)\left(\dv{r}{\tau}\right)^2 + \left(\frac{GM}{r^2}\right)\left(1-\frac{2GM}{r}\right)\left(\dv{t}{\tau}\right)^2 \quad & \, \\
- (r - 2GM)\left[\left(\dv{\theta}{\tau}\right)^2 + \left(\dv{\phi}{\tau}\right)^2\right] & = 0
\end{align*}

It may be messy, but it works. All four of the free variable, choices gave the same value for both geodesic equations.

Copying down the four geodesics provide:
\[\dv[2]{t}{\tau} - \left(1 - \frac{2GM}{r}\right)^{-1}\left(\frac{2GM}{r^2}\right)\dv{x^t}{\tau}\dv{x^r}{\tau} 1= 0\]
\[\dv[2]{x^\theta}{\tau} + \frac{2}{r}\dv{x^r}{\tau}\dv{x^\theta}{\tau} =  0\]
\[\dv[2]{\phi}{\tau} + \frac{2}{r}\dv{\phi}{\tau}\dv{r}{\tau} = 0\]
\begin{align*}
\dv[2]{r}{\tau} + \left(1 - \frac{2GM}{r}\right)^{-1}\left(\frac{2GM}{r}\right)\left(\dv{r}{\tau}\right)^2 & + \left(\frac{GM}{r^2}\right)\left(1-\frac{2GM}{r}\right)\left(\dv{t}{\tau}\right)^2 \\
& - (r - 2GM)\left[\left(\dv{\theta}{\tau}\right)^2 + \left(\dv{\phi}{\tau}\right)^2\right] = 0
\end{align*}

\end{enumerate}

\end{document}
