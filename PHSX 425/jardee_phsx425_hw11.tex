\documentclass[12pt]{article}

\usepackage[english]{babel}

% Math/Greek packages
\usepackage{amssymb,amsmath,amsthm, mathtools} 
\usepackage{algorithm, algorithmic}
\usepackage{upgreek, siunitx}

% Graphics/Presentation packages
\usepackage{geometry, graphicx}
\usepackage{tabulary, enumitem, array}
\usepackage{xparse,mleftright,tikz}
\usepackage{physics}

% Misc packages
\usepackage{fancyhdr}


\usepackage[export]{adjustbox}

\usepackage{esint}

\sisetup{locale=US,group-separator = {,}}
\usepackage[colorlinks=true, allcolors=blue]{hyperref}


% Box function - update this as more sophisticated solutions are found
\newcommand\mybox[2][]{\tikz[overlay]\node[fill=blue!20,inner sep=2pt, anchor=text, rectangle, rounded corners=1mm,#1] {#2};\phantom{#2}}
\renewcommand{\arraystretch}{1.2}

% General macro declarations


\makeatletter
\let\oldabs\abs
\def\abs{\@ifstar{\oldabs}{\oldabs*}}
%
\let\oldnorm\norm
\def\norm{\@ifstar{\oldnorm}{\oldnorm*}}
\makeatother

\begin{document}

\title{PHSX 425: HW11}
\author{William Jardee}
\maketitle

\textbf{See problem 9.27}

\noindent
\emph{In class, we solved the problem of relfection and refraction at oblique incidence to an interface between linear media, for the case of polarization in the plane of incidence.}
\section*{Question 1}
\emph{Derive $E_{0R}$ and $E_{0T}$ for the case of polarization perpendicular to the plane of incidence. Write your answer in terms of $E_{0I}$, $\alpha$, and $\beta$.}

\section*{Question 2}
\emph{Also derive the reflection and transmission coefficients. Show that they sum to unity.}

\section*{Question 3}
\emph{Using the values $n_1 = 1$, $n_2 = 1.5$, with $\mu_1 = \mu_2 = \mu_0$, plot $E_{0R}/E_{0T}$ and $E_{0T}/E_{0I}$ as a function of the incidence angle, $\theta_I$.}

\section*{Question 4}
\emph{Also plot the reflection and transmission coefficients as a function of $\theta_I$.}

\end{document}