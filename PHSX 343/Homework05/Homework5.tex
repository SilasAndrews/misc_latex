\documentclass[11pt]{article}

\usepackage{amssymb,amsmath}
\usepackage{times,psfrag,epsf,epsfig,graphics,graphicx,caption}
\usepackage{enumitem}
\usepackage{algorithm}
\usepackage{algorithmic}

\begin{document}
\date{}

\title{PHSX 343: Assignment 5}

\author{William Jardee}

\maketitle


\section*{Problem 1}

\begin{enumerate}[label=\alph*)]
    \item 
        $(\Delta s)^2 = (c\Delta t)^2 -(\Delta x)^2$ Metric Equation
    \[\Delta t = \frac{\Delta x}{v}\]
    \[(c\Delta s)^2 = \Big(\frac{c\Delta x}{v}\Big)^2 - (\Delta x)^2 \rightarrow \Delta s = \sqrt{\Big(\frac{c \Delta x}{v}\Big)^2 - (\Delta x)^2}\]
    \begin{center}
        Where $\Delta x = 52.4 m$ and $v = 0.800c$, so $c\Delta s = 39.3s$ and $\Delta s = 1.31\times 10^{-7}= 131ns$.
    \end{center}
    The proper time of the muon's frame is also the spacetime for the problem since the muon is inertial and $\Delta x_{muon} = 0$. So $\Delta \tau = 131 ns$.
    
    \item
    \[\Delta t = \frac{\Delta t'}{\sqrt{1-\Big(\frac{v}{c}\Big)^2}}\]
    Where $\Delta t'$ is the proper time and $\Delta t$ is any other coordinate time. So, to solve for $\Delta t'$:
    \[\Detla t' = \frac{\Delta x}{v} \sqrt{1-\Big(\frac{v}{c}\Big)^2} = 1.31\times10^{-7} = 131ns\]
    \newline
    These two values should be the same, as they are. 
    
\end{enumerate}

\section*{Problem 2}

\begin{enumerate}[label=\alph*)]
    \item 
    We can use a binomial expansion on the integrad, giving us 
    \[\sqrt{1-\Big(\frac{v}{c}\Big)^2} = 1-\frac{1}{2}\Big(\frac{v}{c}\Big)^2 + \frac{1}{2}\Big(-\frac{1}{2}\Big)\Big(\frac{v}{c}\Big)^4-\dots\]
    To determine how many terms to keep We have to determine how many decimal places $\Big(\frac{v}{c}\Big)^2$ provides. If we name $\Delta t = 1.00 \times 10^6 s$ and $a = 10\frac{m}{s^2}$, then
    \[\Big(\frac{v}{c}\Big)^2 = \Big(\frac{a\Detla t}{c}\Big)^2 = \Big(\frac{1}{30}\Big)^2 = 0.0011\]
    With this analysis, it is obvious that $0.0003$ to any power greater than 1 will give a value with less than 4 fig sigs, when we are adding to 1. Then out integrad becomes:
    \[1-\Big(\frac{v}{c}\Big)^2\]
    
    \item To integrate the problem we can just double how long it takes to travel from A to B. Similarly we can break the integral into two calculations, one from A to the midpoint, then from the midpoint to B. We have an equation from the description of the problem for the acceleration and can use that for v(t).
    \[\Delta \tau_{A \rightarrow B}= \int_0^{\Delta t} (1-\frac{1}{2}\Big(\frac{at'}{c}\Big)^2dt'+\int_0^{\Delta t} (1-\frac{1}{2}\Big(\frac{a\Delta t - at'}{c}\Big)^2dt'\]
    \[\Delta \tau_{A \rightarrow B} = \Big[t-\frac{1}{6}\frac{c}{a}\Big(\frac{at'}{c}\Big)^3\Big]\Big|_0^{\Delta t} + \Big[t+\frac{1}{6}\frac{c}{a}\Big(\frac{a\Delta t - at'}{c}\Big)^3\Big]\Big|_0^{\Delta t}\]
    \[\Delta \tau_{A \rightarrow B} = \Big[\Delta t-\frac{1}{6}\frac{c}{a}\Big(\frac{a\Delta t}{c}\Big)^3 + \Delta t+\frac{1}{6}\frac{c}{a}\Big(\frac{-a\Delta t}{c}\Big)^3\Big] = 2\Delta t - \frac{1}{3}\frac{a^2}{c^2}\Delta t^3\]
    \[\Delta \tau_{A \rightarrow B} = \Delta t \Big(2-\frac{1}{3}\Big(\frac{a\Delta t}{c}\Big)^2\Big)\]
    To give the total path of $\Delta \tau_{A \rightarrow A}$, we just have to double the the value. The following proper time is for the whole path.
    \[\Delta \tau = 3.9993\times 10^6\]
    Now to find the difference in time, 
    \begin{center}
            $4\Delta t - \Delta \tau = 0.0007 \times 10^6 \rightarrow 0.001 \times 10^6 $ (4 sig figs)
    \end{center}
    
\end{enumerate}

\end{document}
