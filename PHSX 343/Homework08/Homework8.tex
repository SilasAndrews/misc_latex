\documentclass[11pt]{article}

\usepackage{amssymb,amsmath}
\usepackage{times,psfrag,epsf,epsfig,graphics,graphicx,caption}
\usepackage{enumitem}
\usepackage{algorithm}
\usepackage{algorithmic}

\begin{document}
\date{}

\title{PHSX 343: Assignment 8}

\author{William Jardee}

\maketitle


\section*{Problem 1}
    \begin{enumerate}[label=\alph*)]
        \item 
            Using the equation for relativistic energy, $E = \gamma mc^2$, we can solve for the velocity with respect to mass and total energy:
            \[\frac{1}{1-\Big(\frac{u}{c}\Big)^2} = \frac{E}{mc^2}\]
            \[\frac{u}{c} = \sqrt{1-\Big(\frac{mc^2}{E}\Big)^2}\]
            Substituting in $mc^2 = 197 \times 939\times 10^6eV$ and $E = 197 \times 1.00\times 10^{11}eV$, $u = 0.99995591298c$, or $c-u = 0.000044087c=4.41\times 10^5c$. \\
            Using the equation of momentum: $p_x=\gamma mu$, then $p_x = 1.9696\times 10^{13}\frac{eV}{c}$ and $p_t = \gamma mc$ then $p_t = 1.9697 \times  10^{13} \frac{eV}{c}$.
        \item
            Using the translations in class: $p'_t = \gamma (p_t - \beta p_x)$ and $p'_x = \gamma (p_x - \beta p_t)$. The $\gamma$ and $\beta$ come from the frame that gives proper time, a.k.a. the frame of the particle we are moving with, with the same velocity as above. We also know that the two momentum will be moving in opposite directions, so if $P_t$ is positive then $P_x$ is negative. (We can just throw out $P_y$ and $P_z$ because they are both zero)
            \[P'_t = \gamma (P_t + \beta P_x) = 4.19\times 10^{15} \frac{eV}{c}\]
            \[P'_x = \gamma (-P_x -\beta p_t) = -4.19\times 10^{15} \frac{eV}{c}\]
            \\
            For the energy we can use $E^2 = (PC)^2 + (mc^2)^2$, where $P = P_t - P_x - P_y - P_z$.
            \[E = \sqrt{(PC)^2 + (mc^2)^2}= \sqrt{(2*4.19\times 10^{15})^2+ (197*939\times 10^6)^2}\]
            \[= 8.38\times 10^{15}\]
    
    \end{enumerate}

\section*{Problem 2}
    We know from lecture that $cP_t = E$, so using this and conservation of momentum on the initial condition that $E_a+E_b=E_c$:
    \[E_a + E_b = E_c + E_d \rightarrow c(P_{ta} +P_{tb}) = c(P_{tc} +P_{td})\]
    This is equivalent to saying conservation of momentum, and as we know conservation of momentum holds in all frames. 
    \[E'_a + E'_b = cP'_{ta} + cP'_{tb}=c(P'_{ta} + P'_{tb})\]
    \[=c(\gamma(P_{at} - \beta P_{ax} + P_{bt} - \beta P_{bx})) = c(\gamma(P_{at}+ P_{bt})-\gamma \beta (P_{ax} + P_{bx}))\]
    \[= c(\gamma(P_{ct}+ P_{dt})-\gamma \beta (P_{cx} + P_{dx})) = c(\gamma(P_{ct} - \beta P_{cx} + P_{dt} - \beta P_{dx}))\]
    \[= c(P'_{tc} + P'_{td}) = cP'_{tc} + cP'_{td} = E'_c + E'_d\]
    
\section*{Problem 3}
    Since we get the approximation that $K=\frac{1}{2}mu^2$ by removing positive terms off the taylor expansion, the accurate K will always be $K\geq \frac{1}{2}mu^2$. The only time that $K=\frac{1}{2}mu^2$ is when the removed terms are zero, or when $K=0$. This happens when at rest, $u=0$, or it has zero mass, $m=0$, such as with a photon. It can also be said that for lower energies the kinetic energy can be approximated well as $K=\frac{1}{2}mu^2$, but this is only approximately true for $u<0.1c$.
    \begin{enumerate}[label=\alph*)]
        \item $K<\frac{1}{2}mu^2$: Never
        \item $K = \frac{1}{2}mu^2$: Only when $K=0$\\\\
                $K \approx \frac{1}{2}mu^2$: When $u<0.1c$
        \item $K>\frac{1}{2}mu^2$: Whenever $K \neq 0$
    \end{enumerate}

\end{document}
