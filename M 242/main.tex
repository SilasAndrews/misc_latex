\documentclass[10pt] {report}
\usepackage{graphicx, ulem, amsmath, amsfonts, amssymb}
\pagestyle{plain}
  \def\ds{\displaystyle }
  \setlength{\topmargin}{-1 in}
  \setlength{\textheight}{9.5 in}
  \setlength{\oddsidemargin}{-.25 in}
  \setlength{\evensidemargin}{-.25 in}
  \setlength{\textwidth}{6.6 in}
\begin{document}
\begin{center}
Math 242\\
William Jardee\\
LaTex Write Up\\
Chapter 8, Proof 3.\\
\today\\

\end{center}
\begin{enumerate}
\item[2.] For every natural number $n$, the integer $2n^2 -4n+31$ is prime.\\
False.\\
Disproof by Counterexample:\\
Suppose that $n=31$.  Then $2n^2 -4n+31=(31)^2 -4*31 +31= 868=31(28)$. Since the integer 868 is not prime, the statement does not hold for all values of n and is thus false. 

\item[8.] If A, B, and C are sets, then $A-(B \cup C)=(A-B)\cup(A-C).$\\
True.\\
Direct Proof:\\
Suppose A, B, and C are sets. Then $A-(B \cup C)=\{x:x\in A-(B \cup C)\}$ using set builder notation.\\
    $=\{x:x \in A \cap x \notin (B \cup C)\}$ \hfill (Definition of Compliment)\\
    $=\{x:x \in A \cap (x\notin B \cup x\notin C)\}$ \hfill (Definition of Union)\\
    $=\{x:(x \in A \cap x\notin B) \cup (x \in A \cap x \notin C)\}$ \hfill (Distributive Property)\\
    $=\{x:(x \in A-B) \cup (x\in A-C)\}$ \hfill (Definition of Compliment)\\
    $=\{x:x \in (A-B) \cup (A-C)\}$ \hfill (Definition of Union)\\
Rewriting the set in the intentional definition,\\
$=(A-B) \cup (A-C)$\\
Thus we have shown that $A-(B \cup C)=(A-B) \cup (A-C)$ by means of direct proof.
\begin{flushright} QED \end{flushright}
\item[9.] If A and B are sets, then $\mathcal{P}(A)-\mathcal{P}(B)\subseteq\mathcal{P}(A-B)$.\\
False.\\
Disproof by Contradiction:\\
Assume we have sets $A=\{1,2\}$ and $B=\{2,3\}$.\\
Then $\mathcal{P}(A)=\{\{\},\{1\},\{2\},\{1,2\}\}$ and\\
$\mathcal{P}(B)=\{\{\},\{2\},\{3\},\{2,3\}\}$.\\
$\mathcal{P}(A)-\mathcal{P}(B)=\{\{1\},\{1,2\}\}$\\
$A-B=\{1,3\}$, and $\mathcal{P}(A-B)=\{\{\},\{1\},\{3\},\{1,3\}\}$\\
So it follows that $\mathcal{P}(A)-\mathcal{P}(B) \nsubseteq \mathcal{P}(A-B)$.\\
Since $\mathcal{P}(A)-\mathcal{P}(B) \nsubseteq \mathcal{P}(A-B)$, the original statement does not hold true for all possible sets, thus the hypothesis is false.
\item[28.] Suppose $a,b\in\mathbf{Z}$. If $a|b$ and $b|a$, then a=b.\\
False.\\
Disproof by Contradiction:\\
Since the hypothesis includes all integers, we can suppose $a=-1$ and $b=1$. It is obvious that $-1|1$ and $1|-1$. But $a \neq b$, so the hypothesis is false. 
\item[34.] If $X\subseteq A\cup B$, then $X\subseteq A$ or $X\subseteq B$.\\
False.\\
Disproof by Counterexample:\\
Suppose sets A, B, and X such that $A=\{1,2\}$, $B=\{2,3\}$, and $X=\{1,3\}$. The union $A\cup B=\{1,2,3\}$. It is obvious that $X$ is a subset of this union. However, $X\nsubseteq A$ and $X\nsubseteq B$; thus, through the definition of union the hypothesis does not hold true that if $X\subseteq A \cup B$ then $X\subseteq A$ of $X\subseteq B$.
\end{enumerate}
\end{document}