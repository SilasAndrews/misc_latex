%% Document initiation %%
\documentclass[11pt]{article}
\usepackage[utf8]{inputenc}
\usepackage[a4paper, total={6in, 8in}]{geometry}
\setlength{\parskip}{0.5em}


%% Package Declarations %%
\usepackage{amssymb,amsmath, algorithm, algorithmic}
\usepackage{xcolor, times,psfrag,epsf,epsfig,graphics, tabularx, array}
\usepackage{tikz}
\usepackage{multicol, wrapfig, ctable}
\usepackage{fancyhdr, hanging, colortbl, arydshln}


%% Common Declarations %%
\newcommand\mybox[2][]{\tikz[overlay]\node[fill=blue!20,inner sep=2pt, anchor=text, rectangle, rounded corners=1mm,#1] {#2};\phantom{#2}}
\renewcommand{\arraystretch}{1.2}

\pagestyle{fancy}
\begin{document}

%% Document Building %%
\graphicspath{{../images/}}

%% Title %%
\title{CSCI 338: Homework~5~}
\author{William Jardee}
\date{\today}
\maketitle


%% Document Contents %%

\section*{Problem 1}

\noindent
We are given 5 matrices $M_1,...,M_5$, their dimensions (i.e., rows by columns)
are as follows: 
$M_1$ is 15 $\times$ 20,
$M_2$ is 20 $\times$ 30,
$M_3$ is 30 $\times$ 10,
$M_4$ is 10 $\times$ 50, and
$M_5$ is 50 $\times$ 8.
\newline

\noindent
(1.1) Run the dynamic programming algorithm for {\em matrix chain multiplication} that we covered in class to produce the table $m[-,-]$.

\begin{center}
\begin{tabular}{c|c|c|c|c|c|} 
    (i,j)  & 1 & 2 & 3 & 4 & 5 \\ \hline
    1 & 0 & 9000 & 13500 & 21000 & 13600 \\ \hline
    2 & \cellcolor{gray} & 0 & 6000 & 16000 & 11200 \\ \hline
    3 & \cellcolor{gray} & \cellcolor{gray} & 0 & 15000 & 6400\\ \hline
    4 & \cellcolor{gray} & \cellcolor{gray} & \cellcolor{gray} & 0 & 4000\\ \hline
    5 & \cellcolor{gray} & \cellcolor{gray} & \cellcolor{gray} & \cellcolor{gray} & 0\\ \hline
\end{tabular}
\end{center}
The calculations can be seen at the end of this page and onto the next. 
 

\noindent
(1.2) What is the optimal solution value? Where do you find it? 
\newline

Your optimal solution (assuming you want to do the full multiplication of 1 through 5) will the in the final position $(1,5)$. So $m[1,5] =$ \mybox[fill=blue!20]{$13600$}.\\

All that beautiful math that I promised earlier:\\
The diagonal is trivially 0. and the elements above the diagonal will be the product of the edges; i.e. $m[1,2] = 15 \times 20 \times 30 = 9000$

\begin{tabular}{r l l}
    $m[1,3]$    & $= m[1,1] + m[2,3] + 15\times 20\times50$  \quad \quad $m_1(m_2 m_3)$\\
                & $= 0 + 6000 + 15000 = 21000$\\
    \cdashline{2-2}
                & $= m[1,2] + m[3,3] + 15\times30\times10$  \quad \quad $(m_1 m_2) m_3$\\
                & $= 9000 + 0 + 4500 = $ \mybox[fill=gray!20]{$13500$}
\end{tabular}   

\begin{tabular}{r l l}
    $m[2,4]$& $= m[2,2] + m[3,4] + 20\times 30\times50$ & \quad \quad $m_2(m_3 m_4)$\\
    & $= 0 + 15000 + 30000 = 45000$ \\
    \cdashline{2-2}
    & $= m[2,3] + m[4,4] + 20\times10\times50$     &     \quad \quad $(m_2 m_3) m_4$\\
    & $= 6000 + 0 + 10000 = $ \mybox[fill=gray!20]{16000}
\end{tabular}

\begin{tabular}{r l l}
    $m[3,5]$& $= m[3,3] + m[4,5] + 30\times 10\times8$ & \quad \quad $m_3(m_4 m_5)$\\
    & $= 0 + 4000 + 2400 =$ \mybox[fill=gray!20]{6500}\\
    \cdashline{2-2}
    & $= m[3,4] + m[5,5] + 15\times30\times10$ & \quad \quad $(m_3 m_4) m_5$\\
    & $= 15000 + 0 + 1200 = 27000$\\
\end{tabular}

\begin{tabular}{r l l}
    $m[1,4]$& $= m[1,1] + m[2,4] + 15\times 20\times50$ & \quad \quad $m_1(m_2 m_3 m_4)$\\
    & $= 0 + 16000 + 15000 = 31000$\\
    \cdashline{2-2}
    & $= m[1,2] + m[3,4] + 15\times30\times50$ & \quad \quad $(m_1 m_2) (m_3 m_4)$\\
    & $= 9000 + 15000 + 22500 = 46500$\\
    \cdashline{2-2}
    & $= m[1,3] + m[4,4] + 15\times10\times50$ & \quad \quad $m_1 (m_2 m_3 m_4)$\\
    & $= 13500 + 0 + 7500 = $ \mybox[fill=gray!20]{21000}\\
\end{tabular}

\begin{tabular}{r l l}
    $m[2,5]$& $= m[2,2] + m[3,5] + 20\times 30\times8$ & \quad \quad $m_2(m_3 m_4 m_5)$\\
    & $= 0 + 6400 + 4800 =$ \mybox[fill=gray!20]{11200}\\
    \cdashline{2-2}
    & $= m[2,3] + m[4,5] + 20\times10\times8$ & \quad \quad $(m_2 m_3) (m_4 m_5)$\\
    & $= 6000 + 4000 + 1600 = 11600$\\
    \cdashline{2-2}
    & $= m[2,4] + m[5,5] + 20\times50\times8$ & \quad \quad $m_2 (m_3 m_4 m_5)$\\
    & $= 16000 + 0 + 8000 = $ 24000\\
\end{tabular}

\begin{tabular}{r l l}
    $m[1,5]$& $= m[1,1] + m[2,5] + 15\times 20\times8$ & \quad \quad $m_1(m_2 m_3 m_4 m_5)$\\
    & $= 0 + 11200 + 2400 =$ \mybox[fill=gray!20]{13600}\\
    \cdashline{2-2}
    & $= m[1,2] + m[3,5] + 15\times30\times8$ & \quad \quad $(m_1 m_2) (m_3 m_4 m_5)$\\
    & $= 9000 + 6400 + 3600 = 19000$\\
    \cdashline{2-2}
    & $= m[1,3] + m[4,5] + 15\times10\times8$ & \quad \quad $(m_1 m_2 m_3) (m_4 m_5)$\\
    & $= 13500 + 4000 + 1200 = $ 18700\\
    \cdashline{2-2}
    & $= m[1,4] + m[5,5] + 15\times50\times8$ & \quad \quad $(m_1 m_2 m_3 m_4 ) m_5$\\
    & $= 21000 + 0 + 6000 = $ 27000\\
\end{tabular}

\newpage

\section*{Problem 2}

\noindent
We are given a context-free grammar $G$ as follows:
\newline

$G$: $S\rightarrow AS|SB$

~~~  $A\rightarrow AD|DA|a$

~~~  $B\rightarrow BB|BD|b$

~~~  $D\rightarrow DD|d$.

We are also given a string $w=bdbdd$.
\newline

\noindent
(2.1) Run the dynamic programming algorithm for $A_{CFG}$ that we covered in class to produce the table $table[-,-]$.
\newline

\begin{center}
\begin{tabular}{c|c|c|c|c|c|} 
    (i,j)  & 1 & 2 & 3 & 4 & 5 \\ \hline
    1 & S,B & B & S,B & S,B & S,B \\ \hline
    2 & \cellcolor{gray} & D & $\emptyset$ & $\emptyset$ & $\emptyset$ \\ \hline
    3 & \cellcolor{gray} & \cellcolor{gray} & B & B & B\\ \hline
    4 & \cellcolor{gray} & \cellcolor{gray} & \cellcolor{gray} & D & D\\ \hline
    5 & \cellcolor{gray} & \cellcolor{gray} & \cellcolor{gray} & \cellcolor{gray} & D\\ \hline
\end{tabular}
\end{center}

\noindent
(2.2) How do we know whether $G$ generates $w$ from the table?\\

The fact that $m[1,5]$ isn't $\emptyset$ means that we can get to $w$. Furthermore, since I was careful to check where ``$S$" can get me, the fact that ``$S$" is in the $m[1,5]$ position means that we can get to $w$ from our start position. (When I include ``$S$" it means that the cell can be solved by a concatenation of two cells that form a string $BB$.)  


\newpage


\section*{Problem 3}

Show that {\em ALL}$_{DFA}\in$ P.

We can show that {\em ALL}$_{DFA}\in$ P pretty easily if we simplify it to something that we have already analyzed. In the book it is given that {\em PATH} $\in$ P and in that proof it is asserted that a breadth-first search or a depth-first search are both polynomial length processes. 

Build our $DFA$ like a directed graph. Then, starting on our $q_0$, check every path using one of these two searching mechanisms (I suggest breadth-first). If any of these paths pass through at least one invalid node then we reject. Otherwise, if all paths are valid, then accept. Notice how we have a deterministic setup, so every string has a unique node it will land on and the argument only works if you include all nodes (even implicitly stated ones). We have a polynomial number of times that we run a polynomial operation (the number of nodes by the time of a search). Since the product of polynomials is polynomial, the process to check {\em ALL}$_{DFA}\in$ P. 

\newpage

\section*{Problem 4}

Show that Independent Set $\in$ NP.

There are a couple of ways to go about this question, the one I am going to go about uses this core fact outlined in the lecture: 
\begin{center}
    $[V-V^\prime]$ is an IS of $G$\\
    $[V-V^\prime]$ is a cliche of $\overline{G}$
\end{center}

In this case, we will take our graph $G$ and produce $\overline{G}$ (if we use an adjacency matrix this takes time {\em O}$(n^2)$). We know that calculating the cliches of the graph is {\em O}$(n^2)$. Since these two sets are equal, the computation time to verify would be {\em O}$(n^2)$ overall. So IS is NP. 
\newline




\end{document}